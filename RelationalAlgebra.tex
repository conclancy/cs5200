% https://www.overleaf.com/project/640dde6856f1ffc9a91aa9eb

\documentclass{article}
\usepackage{graphicx} % Required for inserting images
\usepackage{listings}
\usepackage{color}
\definecolor{dkgreen}{rgb}{0,0.6,0}
\definecolor{gray}{rgb}{0.5,0.5,0.5}
\definecolor{mauve}{rgb}{0.58,0,0.82}
\lstset{language=SQL,
  basicstyle={\small\ttfamily},
  belowskip=3mm,
  breakatwhitespace=true,
  breaklines=true,
  classoffset=0,
  columns=flexible,
  commentstyle=\color{dkgreen},
  framexleftmargin=0.25em,
  frameshape={}{}{}{},
  keywordstyle=\color{blue},
  showstringspaces=false,
  stringstyle=\color{mauve},
  tabsize=3,
  xleftmargin =1em
}

\title{Relational Algebra \& Calculus Expressions}
\author{Connor Clancy - CS5200 - clancy.co@northeastern.edu}
\date{Spring 2023}

\begin{document}

\maketitle

\begin{enumerate}
    \item \textbf{Find the distinct names of all courses that meet during the "B" block in the "Spring 2023" term.} \[\pi_{cname}(Courses \bowtie_{Courses.cid = Sections.cid} \sigma_{(block = 'B')}(Sections))\]

    \item \textbf{Find the distinct names of all students that took at least nine credits during the "Spring 2023" term.} 
    \[\rho_{SC}(\pi_{sname, hours, sid, term}(Studenst \bowtie Enrollment \bowtie Sections 
    \bowtie Courses))\]
    \[\rho_{CH_{(student, TotalHours)}}(_{<sname>} f _{<SUM_{hours}>} (\sigma_{term='Spring 2023'}(SC))\]
    \[\pi_{student} (\sigma_{TotalHours \ge 9}(CourseHours))\]

    \item \textbf{Find the distinct names of all students who major in either "Accounting" or "Business" and who scored less than 80\% in either course 91.274 or in course 14.102 (using cid).}
    \[\rho_{Sec}(\pi_{sid}(\sigma_{(cid = 91.274) \lor (cid =14.102)}(Sections)))\]
    \[\rho_{En}(\sigma_{grade \ge 80\%}(Sec \bowtie_{Sec.sid = Enrollment.sid} Enrollment))\]
    \[\pi_{sname}(\sigma_{(major = Accounting) \lor (major = Business)}(En 
    \bowtie_{En.tid=Students.tid} Students))\]

    \item\textbf{Find the number of students in each course during the "Spring 2023" term assuming that a course can have several sections.}
    \[\rho_{Spring}(\sigma_{term = Spring 2023}(Sections))\]
    \[\rho_{SS}(Spring \bowtie_{Spring.sid=Enrollment.sid} Enrollment 
    \bowtie_{Enrollment.tid=Students.tid} Students)\]
    \[<cid> f <COUNT_{tid}>(SS)\]

    \item \textbf{How many courses are offered during the "Spring 2023" term that have more than one section?}
     \[\rho_{Spring}(\sigma_{term = Spring 2023}(Sections))\]
     \[\rho_{SecCnt_{cid, sectionCount}}(<cid> f <COUNT_{sid}>(Spring))\]
     \[\rho_{CF}(\sigma_{sectionCount > 1}(SecCnt))\]
     \[<> f <COUNT_{cid}>(CF)\]

     \item \textbf{List the names and majors of all students in the college Khoury who have a GPA below 3.0 and are not on coop.}
     \[\pi_{sname, major}(\sigma_{(college=Khoury) \land (gpa<3) \land (onCoop=False)}(Students))\]

     \item \textbf{Find the distinct names of all courses that have at least 2 but no more than 5 credit hours.}
     \[\{c.cname: Courses(c) \land c.hours \ge 2 \land c.hours \le 5\}\]

     \item \textbf{Write the equivalent tuple relational calculus expression for the SQL statement below:}
     \begin{lstlisting}
      SELECT sname, gpa 
          FROM Students
        WHERE plusOne = T AND gpa < 2.5;
    \end{lstlisting}
    \[\{s.sname, s.gpa: Students(s) \land s.plusOne = T \land s.gpa < 2.5\}\]

    \item \textbf{Write the equivalent relational algebra expression for the SQL statement below:}
    \begin{lstlisting}
        SELECT sname, gpa
            FROM Students
          WHERE plusOne = T 
            AND (gpa BETWEEN 2.99 AND 4.0);
    \end{lstlisting}
    \[\pi_{sname, gpa}(\sigma_{(plustOne = T) \land (gpa \ge 2.99) \land (gpa \le 4)}(Students))\]

    \item \textbf{Write a single equivalent SQL statement for the relational algebra expression below:}
    \[\rho_{KhourySection}(\sigma_{college='Khoury'}(Courses \bowtie Sections))\]
    \[\pi_{term, cname, room}(\sigma_{block='G' \lor block='H'}(KhourySections)\]
    \begin{lstlisting}
        SELECT term, cname, room
        FROM Courses AS c
        INNER JOIN  Sections AS s
            s.cid = c.cid
        WHERE 
            c.college = 'Khoury' 
            AND (s.block = 'G' OR s.block = 'H')
    \end{lstlisting}
\end{enumerate}

\end{document}
