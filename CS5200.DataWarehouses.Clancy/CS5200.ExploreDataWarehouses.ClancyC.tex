% Options for packages loaded elsewhere
\PassOptionsToPackage{unicode}{hyperref}
\PassOptionsToPackage{hyphens}{url}
%
\documentclass[
]{article}
\usepackage{amsmath,amssymb}
\usepackage{lmodern}
\usepackage{iftex}
\ifPDFTeX
  \usepackage[T1]{fontenc}
  \usepackage[utf8]{inputenc}
  \usepackage{textcomp} % provide euro and other symbols
\else % if luatex or xetex
  \usepackage{unicode-math}
  \defaultfontfeatures{Scale=MatchLowercase}
  \defaultfontfeatures[\rmfamily]{Ligatures=TeX,Scale=1}
\fi
% Use upquote if available, for straight quotes in verbatim environments
\IfFileExists{upquote.sty}{\usepackage{upquote}}{}
\IfFileExists{microtype.sty}{% use microtype if available
  \usepackage[]{microtype}
  \UseMicrotypeSet[protrusion]{basicmath} % disable protrusion for tt fonts
}{}
\makeatletter
\@ifundefined{KOMAClassName}{% if non-KOMA class
  \IfFileExists{parskip.sty}{%
    \usepackage{parskip}
  }{% else
    \setlength{\parindent}{0pt}
    \setlength{\parskip}{6pt plus 2pt minus 1pt}}
}{% if KOMA class
  \KOMAoptions{parskip=half}}
\makeatother
\usepackage{xcolor}
\usepackage[margin=1in]{geometry}
\usepackage{graphicx}
\makeatletter
\def\maxwidth{\ifdim\Gin@nat@width>\linewidth\linewidth\else\Gin@nat@width\fi}
\def\maxheight{\ifdim\Gin@nat@height>\textheight\textheight\else\Gin@nat@height\fi}
\makeatother
% Scale images if necessary, so that they will not overflow the page
% margins by default, and it is still possible to overwrite the defaults
% using explicit options in \includegraphics[width, height, ...]{}
\setkeys{Gin}{width=\maxwidth,height=\maxheight,keepaspectratio}
% Set default figure placement to htbp
\makeatletter
\def\fps@figure{htbp}
\makeatother
\setlength{\emergencystretch}{3em} % prevent overfull lines
\providecommand{\tightlist}{%
  \setlength{\itemsep}{0pt}\setlength{\parskip}{0pt}}
\setcounter{secnumdepth}{-\maxdimen} % remove section numbering
\ifLuaTeX
  \usepackage{selnolig}  % disable illegal ligatures
\fi
\IfFileExists{bookmark.sty}{\usepackage{bookmark}}{\usepackage{hyperref}}
\IfFileExists{xurl.sty}{\usepackage{xurl}}{} % add URL line breaks if available
\urlstyle{same} % disable monospaced font for URLs
\hypersetup{
  pdftitle={Assignment / Explore Data Warehoues},
  pdfauthor={Connor Clancy - clancy.co@northeastern.edu},
  hidelinks,
  pdfcreator={LaTeX via pandoc}}

\title{Assignment / Explore Data Warehoues}
\author{Connor Clancy -
\href{mailto:clancy.co@northeastern.edu}{\nolinkurl{clancy.co@northeastern.edu}}}
\date{Spring 2023}

\begin{document}
\maketitle

\hypertarget{question-1}{%
\subsection{Question 1}\label{question-1}}

\emph{Data warehouses are often constructed using relational databases.
Explain the use of fact tables and star schemas to construct a data
warehouse in a relational database. Also comment on whether a
transactional database can and should be used to OLAP.}

Fact tables and Star Schemas can be used in an \emph{Online Analytical
Processing (OLAP)} environment to contruct a relational database that is
optimized for analytical querying rather than the traditional
\emph{Online Transaction Processing (OLTP)} which is optimized for
transactional querying. Another way to think about this is that OLAP
databases are optimized for read queries where as OLTP databases are
optimized for write queries. Both environments can be created in
traditional RDBS programs, the main difference is how the tables and
schemas are designed.

In OLAP, fact tables usually consist of pre-calculated and aggregated
data points in a non-normalized schema. This helps to reduce query time
and complexity for the analytical user and allows them to answer
questions about the data in a minimal amount of time. Dimension or `dim'
tables are then set up around the fact table to layer in additional data
not included in the fact table. Star databases usually only require
one-layer of joins from the fact table to get to a dimension table. If
additional layers are added, this is called a snowflake design and can
sometimes be appropriate, but can also slow down processing time for
queries the more complicated these schemas get.

Traditional OLTP databases can be used for analyzing data, but it is
likely that this would be painful for an analytical users who need to
use this databases as their queries would take a long time to develop
(due to the complex nature of the schema they are working work) and the
query would take a long time to execute as compared to the same data in
an OLAP Star Schema database.

\hypertarget{question-2}{%
\subsection{Question 2}\label{question-2}}

\emph{Explain the difference between a data warehouse, a data mart, and
a data lake. Provide at least one example of their use from your
experience or how you believe they might be used in practice. Find at
least one video, article, or tutorial online that explains the
differences and embed that into your notebook.}

Data lakes are the least curated and least structured analytical data
store; they tend to be raw collections of data that are being
centralized from their original sources. My company uses S3 storage
buckets as a data lake to act as a data centralization point entering
our cloud environment. Here you can find JSONs, CSVs, Parquete, and
database table files.

Data warehouses exist after an Extract, Transform, \& Load process
(ETL). This process involves \emph{extracting} data from its source
(either the data lake, or possible another source), \emph{transforming}
the data to match the schema of the data in the data warehouse, and then
\emph{loading} the data into the dataw arehouse tables. My company uses
Hadoop to do the ETL and datawarehousing processes. Once the data ETL'd
it can be accesses using Hive SQL commands by our data engineering and
data management teams.

Data marts are the most curated data stores we are covering in this
assignment. They are subsets of a data warehouse used by a specific
domain (marketing, finance, security, etc.). My company uses Snowflake
to house our data marts; our data marts serve two purposee. The first is
to simplify the domain knowledge requirements for analysts, and the
second is to ensure that we are only exposing analysts to the data
necessary to do their jobs as a security and privacy control.

\hypertarget{question-3}{%
\subsection{Question 3}\label{question-3}}

\emph{After the general explanation of fact tables and star schemas,
design an appropriate fact table for Practicum I's bird strike database.
Of course, there are many fact tables one could build, so pick some
analytics problem and design a fact table for that. Be sure to explain
your approach and design reasons.}

\end{document}
